\documentclass[a4paper, 9pt, twoside]{article}
\usepackage[inner=0.2in, outer=0.2in, top=0.2in, bottom=0.6in]{geometry}
\geometry{papersize={99mm, 210mm}}        % 定义 DL 尺寸
\usepackage[fontset=fandol]{ctex}
% \setCJKmainfont{Noto Serif SC}
\setCJKmainfont{Noto Sans SC}
% \setCJKsansfont{Noto Sans SC}
% \setCJKmonofont{Noto Sans Mono SC}
\usepackage{fontspec} % Enables font customization
\newfontfamily\emojifont{Noto Color Emoji}[Renderer=HarfBuzz]
\setmainfont{Noto Sans SC}
\usepackage{titlesec}
\titleformat{\section}{\vspace{-1.75ex}\centering\Large\bf}{}{0em}{}
\titleformat{\subsection}{\vspace{-1.75ex}\bf}{}{0em}{}[\hrule height 0.5pt\vspace{0ex}]
\usepackage{tocloft} % 控制目录样式
\renewcommand{\contentsname}{} % 清空目录标题
\setlength{\cftbeforetoctitleskip}{0pt} % 移除标题前的空白
\setlength{\cftaftertoctitleskip}{0pt}  % 移除标题后的空白
\setcounter{tocdepth}{1} % 显示到 \section 层级

\usepackage{chat-history}

% 逗哏的人(出现在左侧)
\newchatperson[gray]{jg}{avatar/jg.jpg}{花椒果汁}
\newchatperson[gray]{sm}{avatar/default.jpg}{Simone-IARP中国联络员}
\newchatperson[gray]{dn}{avatar/default.jpg}{Dan}
\newchatperson[gray]{nw}{avatar/default.jpg}{星光立交与星河路互通匝道}
\newchatperson[gray]{li}{avatar/default.jpg}{雷狼狼}
\newchatperson[gray]{sh}{avatar/default.jpg}{舒扒皮}
\newchatperson[gray]{aml}{avatar/default.jpg}{阿猫的猫}
\newchatperson[gray]{bdl}{avatar/default.jpg}{白貂Akirua}
\newchatperson[gray]{xf}{avatar/default.jpg}{雪风:欢迎来到现实的荒漠}

% 捧哏的人(出现在右侧)
\newchathost[RoyalBlue]{cx}{avatar/default.jpg}{咸狼赤弦}
\newchathost[gray]{wm}{avatar/default.jpg}{{\emojifont🐾}神前 \ 海/彦烈{\emojifont🐾}}
\newchathost[gray]{bdr}{avatar/default.jpg}{白貂Akirua}
\newchathost[gray]{nl}{avatar/default.jpg}{纳尔}
\newchathost[gray]{gf}{avatar/default.jpg}{光锋-ZH9c418}
\newchathost[gray]{lr}{avatar/default.jpg}{雷尔想吃电弧星}
\newchathost[gray]{hq}{avatar/default.jpg}{草莓绿茶曲奇}
\newchathost[gray]{amr}{avatar/default.jpg}{阿猫的猫}
\newchathost[gray]{xz}{avatar/default.jpg}{【吃瓜】棕汪}


\begin{document}

% 封面

\pagenumbering{gobble} % 禁用页码
\begin{titlepage}
    \centering
    \vspace*{2cm}
    \begin{tikzpicture}[baseline=(current bounding box.north)]
        \clip (0,0) circle(2cm); % Clip the avatar image to a circle
        \node[anchor=center,inner sep=0pt,minimum size=1cm] (avatar) at (0,0)
        {\includegraphics[width=4cm,height=4cm,keepaspectratio]{avatar/group.jpg}};
    \end{tikzpicture}\\[1cm]
    {\Huge\bfseries ``弦狼实验室''}\\[0.5cm]
    {\Huge\bfseries 年度冷笑话合订本}

    \vfill
    截至\today
\end{titlepage}

% 前言

\pagenumbering{roman} % 页码重置为阿拉伯数字
\setcounter{page}{1}   % 从第1页开始编号

\section{主要贡献者}
{\vspace{-2ex} \centering \small(按第一次出现时间排序) \par}

\begin{tikzpicture}[baseline=(current bounding box.north)]
    \avatarWithText{0}{0}{avatar/jg.jpg}{花椒果汁}
    \avatarWithText{0}{-0.8cm}{avatar/default.jpg}{Simone-IARP中国联络员}
    \avatarWithText{0}{-1.6cm}{avatar/default.jpg}{Dan}
    \avatarWithText{0}{-2.4cm}{avatar/default.jpg}{星光立交与星河路互通匝道}
    \avatarWithText{0}{-3.2cm}{avatar/default.jpg}{雷狼狼}
    \avatarWithText{0}{-4.0cm}{avatar/default.jpg}{舒扒皮}
    \avatarWithText{0}{-4.8cm}{avatar/default.jpg}{阿猫的猫}
    \avatarWithText{0}{-5.6cm}{avatar/default.jpg}{白貂Akirua}
    \avatarWithText{0}{-6.4cm}{avatar/default.jpg}{雪风:欢迎来到现实的荒漠}
\end{tikzpicture}
    
\newpage

\section{目录}
\tableofcontents % 添加目录

\newpage

% content

\pagenumbering{arabic} % 页码重置为阿拉伯数字
\setcounter{page}{1}   % 从第1页开始编号

\section{2024年}

\subsection{5月16日}
\jg{话说人摸电线会触电,那么电线被人摸就会触手?}

\subsection{5月17日}
\jg{为什么虹桥机场不能去阿斯加德?}

\subsection{6月3日}
\context{在讨论JWST不能观测全部的可见光波长}
\cx{主要是观测红外 \ 从600纳米到25微米}
\jg{@咸狼赤弦 \ 不可见光的东西,要好好查查他们的账本}

\subsection{6月7日}
\context{在讨论美国光学望远镜建造部门的反间谍措施}
\jg{蠢人容易被情报人员利用,是因为intelligence被偷走了吗}

\subsection{6月15日}
\jg{如果湘菜里有芹菜,那秦菜里有香菜吗?}

\subsection{6月20日}
\context{在讨论飞机上生的小孩该起什么名字}
\jg{“你是天生的” “天生什么的?” “就是天生的”}

\subsection{6月27日}
\context{赤弦误入了几个天文系教授的聊天后错过了开溜时机}
\cx{不过真要这么说的话我觉得他们应该是在劝退}
\cx{一直在讲学术圈的黑暗}
\jg{(毕竟有阳光照到的地方研究不了天文}

\subsection{6月30日}
\jg{大部分人都用右手写字是不是因为这才是right hand?}

\subsection{7月4日}
\context{在讨论购买键盘时的考虑事项}
\wm{糖进键盘, WD40和酒精都救不了}
\jg{@{\emojifont🐾}神前 \ 海/彦烈{\emojifont🐾} \ 所以不要使用糖特别多的语言}

\subsection{7月8日}
\context{一张小英雄其实很强的梗图}
\jg{@咸狼赤弦 \ 小英雄不是触手可得的}

\subsection{7月9日}
\bdr{为什么黄石公园不卖硫磺皂周边}
\jg{黄石坐高铁去武汉,然后坐地铁去柏林,这是美国到欧洲最快的路线}

\subsection{7月11日}
\context{在讨论山东宾果游戏中的刻板印象}
\nl{我172}
\cx{可窝认识的山东人均大汉}
\jg{@咸狼赤弦 \ 陕西人均大唐?}

\subsection{7月19日}
\context{在隔壁群讨论ArXiv中Cosmology分类}
\cx{好奇怪}
\cx{用google app打开arxiv 朗读 结果读出来的是8天前的页面的内容}
\jg{研究cosmology的人叫coser吗?}

\subsection{7月21日}
\context{在讨论未来规划}
\cx{窝开卷是为了以后也能像现在这样到处玩}
\sm{对cxjj来说就没有闭卷考试,全都是开卷}

\subsection{7月21日}
\context{在讨论推特上偶遇的天文学小动物的研究方向}
\cx{他觉得mit是个好地方}
\cx{那他肯定是做系外行星的 猜中了.jpg}
\jg{@咸狼赤弦 \ 好厉害,随手搓个行星}

\subsection{7月22日}
\jg{愚公移山是个人行为,所以是于私移山}

\subsection{7月23日}
\jg{话说……冒脑花是不是一种烫头}
\jg{只不过烫的是里面}

\subsection{7月23日}
\context{知乎问题``看完奥本海默突然想学理论物理可行吗''}
\gf{@咸狼赤弦 \ 看完黑客帝国我想学功夫躲子弹可以吗}
\jg{@光锋-ZH9c418 \ 不不不,那是kongfu不是功夫,意思是你学完只能饿肚子喝西北风}

\subsection{7月28日}
\jg{地铁和公交车相比最大的缺点是,它不巴适}

\subsection{7月30日}
\context{知乎问题``空间和时间是有质量的吗?''}
\jg{时间当然是有质量的,物理学家会把质量小的年份叫做light year。}

\subsection{8月1日}
\dn{如果基督教里666代表邪恶 是不是意味着25.8是邪恶的根}

\subsection{8月5日}
\context{看到一个好看的画师}
\cx{想约}
\jg{@咸狼赤弦 \ 为什么要约荤的呢,因为素的不可约}

\subsection{8月5日}
\context{在讨论雷尔做毛替的动作和本体出毛不太像}
\cx{被tf的不够深度}
\lr{来点深度tf}
\jg{@雷尔想吃电弧星 \ 深度tf就是cuda?}
\jg{不对tf本来就是用来深度的啊}
\jg{transfur的话,也是一种homomorphism}
\jg{额异性不算}
\jg{至少确实morph了}

\subsection{8月10日}
\context{一张内容为``我不管,取我虎符号令百万雄师''的梗图}
\jg{为啥虎符可以号令雄狮?}

\subsection{8月11日}
\jg{为什么找零都是大于零元才找?}

\subsection{8月17日}
\context{赤弦在发午饭的照片}
\cx{大块肋眼牛排!}
\jg{@咸狼赤弦 \ 只有刑天才有真正的肋眼}

\subsection{8月18日}
\jg{凭啥四川火锅不算有机食品}
\jg{涮的都是各种内脏}
\jg{各种内脏下水难道不organic吗?}
\nw{草}
\nw{你这有机火锅还是无肌啊}

\subsection{8月19日}
\jg{过敏性肤质会麸质过敏吗?}

\subsection{8月19日}
\li{“我今年实岁26,虚岁28”王警官说着在蛋糕上点了(26+28i)根蜡烛}
\hq{@雷狼狼 其实这个还是比较容易}
\hq{因为蛋糕和C是同构的}
\hq{只要位置找对}
\jg{这叫同胚,同的是蛋糕胚}

\subsection{8月20日}
\context{赤弦发了一张猞猁标本的照片}
\jg{@咸狼赤弦 \ 很多仙侠小说人名喜欢叫啥啥子}
\jg{那猞猁成精了就是猞猁子}

\subsection{8月20日}
\context{在讨论各地小学的师资}
\xz{@阿猫的猫 \ 不是的,至少那个小孩的老师还没我这个县城出身的小学老师讲的明白}
\cx{班主任天天讲什么 剖腹产出来的学生就是不行}
\aml{是不是减负导致的}
\sh{@阿猫的猫 \ 是剪腹导致的}

\subsection{8月26日}
\jg{给没心没肺的人做心肺复苏有用吗?}

\subsection{8月26日}
\context{赤弦在从晒了一天的车里往外搬柜子}
\cx{这几个柜子被拷到烫手的根本拿不起来}
\amr{@咸狼赤弦 \ 用锟斤拷拿出来}

\subsection{8月27日}
\jg{二郎神是不是因为封号太多要念很久,所以叫贯口大神?}

\subsection{8月29日}
\context{赤弦在感慨人生中的错过}
\bdl{你的人生一错再错}
\bdl{所以你应该和我去西藏}
\bdl{去看纳木错、羊卓雍错、玛旁雍措、拉姆拉措…}

\subsection{8月31日}
\jg{燃气灶是不是就是厨师机?}

\subsection{9月6日}
\jg{和尚偷懒睡回笼觉,是不是一种叫回龙观的修行?}

\subsection{9月6日}
\context{在讨论搬到山里后就会变得喜欢发风景照}
\xf{可是}
\xf{我从小就住在山里}
\xf{初中直接沿山而建,体育课项目都是爬山}
\xf{后面知道这山也是风景区}
\xf{所以我是爬山狐}









% last page

\newpage
\pagenumbering{gobble} % Disable page numbering

\null
\vfill

\begin{quote}
    \centering
    访问 \input{ |git remote get-url origin} 获取本文的 \LaTeX 代码. 
\end{quote}

\end{document}
