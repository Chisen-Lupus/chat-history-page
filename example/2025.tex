% 2025.tex 

\section{2025}

\subsection*{1月1日}
\lil{动物园抽血监测 轮到大象的时候大家都喊我}
\lil{因为我负责抽象}

\subsection*{1月1日}
\jgl{firefighters是不是裁军的意思?}

\subsection*{1月1日}
\context{阿猫的猫想起发MFF的图了}
\amr{草}
\lsl{@阿猫的猫 \ 阿猫的猫草}

\subsection*{1月2日}
\context{在讨论晒车}
\asl{我只能}
\asl{晒太阳}
\nwl{我}
\nwl{车没痛}
\nwl{所以不晒了}
\jml{*晒羊毛*}
\lrl{车太黑了,再晒就要被抓去种棉花了}
\jjl{没事,你可以晒鑫牌太阳}
\asl{学历有水分,所以只需要将其晒干即可}

\subsection*{1月2日}
\context{在讨论小时候的麦当劳味道再也没回来过}
\cxr{窝的话}
\cxr{自从吃过了美国的麦当劳肯德基后}
\cxr{就越发想念中国麦当劳惹}
\xfl{@晴天赤弦 \ 想你了,劳大}

\subsection*{1月2日}
\jgl{- commission closed \\ - may i consider it a complement?}

\subsection*{1月4日}
\context{在讨论赤弦的塔罗牌主题的委托}
\lmr{奖牌和法环好像}
\cxr{其实是塔罗牌的星币}
\jgl{@晴天赤弦 \ 说起来starbucks好像就是星币}

\subsection*{1月5日}
\jgl{情侣约会的地方是不是叫七夕地?因为生育率着急所以要保护七夕地}
\jml{我看这个生育率是有乞巧的}

\subsection*{1月6日\footnote{由于统计时的中美时差, 本文标题记录的日期可能会存在不超过一天的误差.}}
\xfl{今天是2024年的头七}

\subsection*{1月6日}
\dnl{飞机动车速度都那么快,那为什么非机动车道限速才15?}

\subsection*{1月7日}
\asl{维京人会戴围巾吗}

\subsection*{1月7日}
\cxr{busybox是不是一种忙盒}
\jjl{是个恐怖盲盒,拆出了Kernel panic}

\subsection*{1月12日}
\jgl{三亚坑游客钱是不是因为那里都是穷B?}

\subsection*{1月13日}
\context{白貂收到了一封标题写着[FREE FOOD]的会议邮件}
\bdr{很懂嘛}
\xzl{@白貂Akirua \ 当学术大拿,在学术会议上大拿特拿}

\subsection*{1月13日}
\context{赤弦明年去凤凰城开会又会遇上兽展}
\cxr{西雅图是ANW 凤凰城是PDFC}
\bdr{你们去凤凰城有点亏啊}
\bdr{这么近}
\bdr{不开会也能去}
\cxr{@白貂Akirua \ 可以开车去 这不省事多了嘛}
\xzr{@晴天赤弦 \ 不能借机旅游呀}
\xzr{我这几年的旅游都是蹭出差去的}
\sml{@棕汪 \ 借机旅游不如包机旅游}

\subsection*{1月15日}
\jgl{听说日本都是日麻不是川麻,感觉那边麻婆豆腐不正宗}

\subsection*{1月16日}
\context{一张兔子的艺术照是抓住兔子拍的}
\lml{这就是抓拍}

\subsection*{1月15日}
\jgl{将会被帅死,帅会被将死}
\jgl{将死之人获得了比赛胜利(}

\subsection*{1月16日}
\dlr{赤弦和余弦有什么区别}
\cyl{@丹尼鹿 \ 赤是正色之一}
\cyl{所以赤弦是一种正弦()}

\subsection*{1月17日}
\jgl{话说}
\jgl{我发现}
\jgl{人行道是人步行道}

\subsection*{1月18日}
\context{赤弦发现FC的万豪酒店的灯的M灭了一半}
\cxr{所以到底什么是λarriott}
\jmr{larriott??}
\npl{@晴天赤弦 \ 五千豪}

\subsection*{1月19日}
\context{赤弦邀请龙羽(夏目龙羽/小毛龙)加入了群聊}
\qsr{xmly}
\cxr{xml.xml}
\xml{其实是}
\xml{xml.yaml}
\xml{now you know Y}

\subsection*{1月22日}
\jgl{话说冬天冷的话,胶兽会不会过粘?}
\yyl{需要降低胶兽的标号}
\jgl{啊但是美国就不会过粘吗}
\cxr{@花椒果汁 (挠头}
\jgl{@晴天赤弦 \ 没事你可以去唐人街,糖人也会过粘}
\jml{不过胶兽总会变成美式的}
\jml{我们这习惯吃粘液饭}
\jgl{但是人家会放鞭炮吓走粘兽}
\jml{不过现在才小粘}
\jml{(?)}
\lml{有没有吃胶子}

\subsection*{1月23日}
\context{雪风发了一张神经网络的结构图汇总}
\xfl{我要发神经了}

\subsection*{1月24日}
\yyl{炸过耗儿鱼的油能不能叫耗儿油}

\subsection*{1月24日}
\jgl{话说}
\jgl{为啥线性代数里没告诉我怎么撩人?}
\jgl{我听人说里面有mate tricks来着?}

\subsection*{1月24日}
\jgl{话说}
\jgl{把拼多多叫成拼夕夕就像把C\#叫成C++}

\subsection*{1月25日}
\context{赤弦从中国经过日本去美国开学后发现老板回国过年了}
\cxr{他说他会先去日本然后去中国}
\qsr{逆向赤弦}
\xfr{《那你呢》}
\qsr{TENET}
\jgl{@棋山 \ 清闲}
\qsr{这下清闲了笑死}
\xfr{钳形行动}
\cxr{真好啊 还有人记得窝有一个叫青弦的设定}

\subsection*{1月25日}
\context{群人数达到了251人}
\cxr{251人了!}
\cxr{“向盟约起誓,在这251秒赌上我的全部!”}
\qsr{什么checkmate宣言}
\aml{@棋山 \ 小动物的生活就是一种checkmate}
\aml{因为check完就mate}

\subsection*{1月27日}
\context{在讨论开车和拍照}
\xfl{同时踩住刹车和油门可以截图}

\subsection*{1月27日}
\context{棋山发了一张正月剃头的梗图}
\qsr{正月里剃头死舅舅\\ 正月接发,舅舅复活。\\ 正月染发,舅舅五彩缤纷。 \\ 正月蛋白矫正,舅舅滑溜溜。 \\ 正月拉直,舅舅变成直男。 \\ 正月烫发,舅舅变成同性恋。}
\mkl{@棋山 \ 正月头发被煤气灶点了,舅舅变成二流子}
\mkl{因为二硫键}

\subsection*{1月28日}
\jgl{话说}
\jgl{鬼灭之刃里为啥是上弦的鬼比较厉害?}
\jgl{难道不是电动的更厉害吗}

\subsection*{1月29日}
\jgl{做人要有同理心,当你数学做不出来的时候,你要想想如果你是高斯、欧拉,你会怎么做}

\subsection*{1月29日}
\lpr{完了 天一黑从地下室走出来迷路了}
\lpr{我是谁我在哪}
\lpr{我要怎么出去}
\nwl{你是路沛}
\nwl{随便走走就是路}
\lpr{那我晚饭吃什么}
\nwl{@陆佩 \ 鲈配}
\aml{@陆佩 \ 鹿醅}
\lpr{我操了}
\lpr{骑摩托的别乱来啊}
\jjl{@阿猫的猫 满脑子都是台湾新闻的 陆配}
\nwl{路赔}
\lsl{露醅}
\lpr{你妈的七八辆摩托车从我四面八方超车}
\lpr{轰油门}
\lpr{吓得我动都不敢动}
\lpr{还有借非机动车道超车的}
\nwl{@陆佩 \ 辘犻}

\subsection*{1月29日}
\xfl{如果python到了3.14版本,是不是可以叫πthon}
\nwl{@雪风:欢迎来到现实的荒漠 \ 但是Python 2.7的时候,没人叫他Ethon啊}

\subsection*{1月30日}
\jgl{死后成仙是一种飞升即走}

\subsection*{1月30日}
\context{赤弦在警察局里排大队很无聊, 整理冷笑话集打发时间}
\cxr{首先恭喜莱森完成了在冷笑话大全中的第一次出现}
\lsl{毕竟我之前是暖男}

\subsection*{1月30日}
\cxr{感觉1月一个月的冷笑话都快抵上一个季度的了}
\aml{@晴天赤弦 \ 冷笑话膨胀是全群变暖导致的吗}

\subsection*{1月30日}
\context{赤弦发现天上有附近空军基地慢悠悠飞出来的飞机}
\cxr{民航一般不会从这里经过}
\lsr{Tucson空域挺干净的}
\lsr{没啥民航}
\lsr{都是军机}
\jgl{军机处}
\jgl{@莱森 \ 不像上海,有专门的闵行区}

\subsection*{1月31日}
\xml{枪毙掉成绩最差的学生可以提高全班平均分}
\xml{这是否是一种few shot learning}

\subsection*{2月1日}
\context{一张"Tomato fucks Egg!"得梗图}
\nwl{番茄 \ 炒 \ 蛋}
\nwl{番茄焯蛋}

\subsection*{2月2日}
\mkr{哪个高等文明发明了熵税}
\qsl{治熵税}
\qsl{我们的宇宙其实是一种汽车电瓶,热力学第二定律其实是因为被抽走供电了}

\subsection*{2月2日}
\context{在讨论赤弦家附近的地貌}
\cxr{不过我家门口的山是那种}
\cxr{岩浆入侵的}
\hgl{@晴天赤弦 \ magma入侵总比maga入侵好}

\subsection*{2月3日}
\jgl{说起来牛头人诞生的故事好像非常牛头人}
\jgl{另外始作俑者是不是对克里特国王说过“啊,米诺斯”}

\subsection*{2月3日}
\context{在讨论各地的蚊子治理}
\bdr{说起来最近蚊子开始入侵新疆了}
\bdr{我这个暑假在伊犁,本地人都说没有蚊子}
\bdr{结果被叮的一塌糊涂}
\aml{@白貂Akirua \ 叮胴急}

\subsection*{2月4日}
\context{在讨论管理群的方法}
\cxr{记得以前有人发现一个人被禁言之后还能当群主 但不能取消自己的禁言 这个bug被修好之前还能君主立宪}
\yxr{@晴天赤弦 \ 群君主立宪制 但是学的是泰国}
\npl{@夜星 \ 君主离线}

\subsection*{2月4日}
\jgl{听说英法控制不了美洲殖民地,导致美国独立,这就叫距离产生美}

\subsection*{2月4日}
\context{赤弦在截图介绍美国银行APP}
\cxr{以防有群友没见过美国银行app的拍照存支票功能}
\lsl{@晴天赤弦 \ 拍照的时候要不要喊:1,2,3,Chase!}

\subsection*{2月5日}
\context{一张介绍把两块木板成直角连接的``抄手榫''的动图}
\jgl{老祖宗的智慧,现在人发明了个新词,叫vlog}
\jgl{虽然这个方向看着是Λlog}

\subsection*{2月7日}
\context{万相遇到了听口音定价的早餐店}
\wxr{甚至还有,糖三角要3块一个,但是你说要糖角(jia)就买一块}
\cyl{@万相 \ 一个糖角一块钱,三个角可不就是三块钱嘛}

\subsection*{2月8日}
\context{夜星在分享家里做的火锅}
\yxr{北美小留特色火锅(大乱炖)}
\qsr{诶}
\qsr{其实也挺好的}
\qsr{有说法是国内那么多火锅类型说到底都是滚水煮万物}
\jgl{@棋山 \ 所以一切都是烧开水}
\yxr{@花椒果汁 \ 寿喜锅、部队锅、火锅}
\jgl{火锅其实是水锅}
\yxr{都算是东亚家庭特色同时处理大量食材的方式(?}
\jgl{处理大量能量的方式}
\yxr{(或者说把一堆东西扔锅里煮一下吃应该算是跨文明和语言的吃法?)}
\jgl{火锅是人类最早的,通过烧开水来转化能量的方式}