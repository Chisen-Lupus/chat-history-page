% 2025.tex 

\section{2025}

\subsection*{1月1日}
\lil{动物园抽血监测 轮到大象的时候大家都喊我}
\lil{因为我负责抽象}

\subsection*{1月1日}
\jgl{firefighters是不是裁军的意思?}

\subsection*{1月1日}
\context{阿猫的猫想起发MFF的图了}
\amr{草}
\lsl{@阿猫的猫 \ 阿猫的猫草}

\subsection*{1月2日}
\context{在讨论晒车}
\asl{我只能}
\asl{晒太阳}
\nwl{我}
\nwl{车没痛}
\nwl{所以不晒了}
\jml{*晒羊毛*}
\lrl{车太黑了,再晒就要被抓去种棉花了}
\jjl{没事,你可以晒鑫牌太阳}
\asl{学历有水分,所以只需要将其晒干即可}

\subsection*{1月2日}
\context{在讨论小时候的麦当劳味道再也没回来过}
\cxr{窝的话}
\cxr{自从吃过了美国的麦当劳肯德基后}
\cxr{就越发想念中国麦当劳惹}
\xfl{@晴天赤弦 \ 想你了,劳大}

\subsection*{1月2日}
\jgl{- commission closed \\ - may i consider it a complement?}

\subsection*{1月4日}
\context{在讨论赤弦的塔罗牌主题的委托}
\lmr{奖牌和法环好像}
\cxr{其实是塔罗牌的星币}
\jgl{@晴天赤弦 \ 说起来starbucks好像就是星币}

\subsection*{1月5日}
\jgl{情侣约会的地方是不是叫七夕地?因为生育率着急所以要保护七夕地}
\jml{我看这个生育率是有乞巧的}

\subsection*{1月6日\footnote{由于统计时的中美时差, 本文标题记录的日期可能会存在不超过一天的误差.}}
\xfl{今天是2024年的头七}

\subsection*{1月6日}
\dnl{飞机动车速度都那么快,那为什么非机动车道限速才15?}

\subsection*{1月7日}
\asl{维京人会戴围巾吗}

\subsection*{1月7日}
\cxr{busybox是不是一种忙盒}
\jjl{是个恐怖盲盒,拆出了Kernel panic}

\subsection*{1月12日}
\jgl{三亚坑游客钱是不是因为那里都是穷B?}

\subsection*{1月13日}
\context{白貂收到了一封标题写着[FREE FOOD]的会议邮件}
\bdr{很懂嘛}
\xzl{@白貂Akirua \ 当学术大拿,在学术会议上大拿特拿}

\subsection*{1月13日}
\context{赤弦明年去凤凰城开会又会遇上兽展}
\cxr{西雅图是ANW 凤凰城是PDFC}
\bdr{你们去凤凰城有点亏啊}
\bdr{这么近}
\bdr{不开会也能去}
\cxr{@白貂Akirua \ 可以开车去 这不省事多了嘛}
\xzr{@晴天赤弦 \ 不能借机旅游呀}
\xzr{我这几年的旅游都是蹭出差去的}
\sml{@棕汪 \ 借机旅游不如包机旅游}

\subsection*{1月15日}
\jgl{听说日本都是日麻不是川麻,感觉那边麻婆豆腐不正宗}

\subsection*{1月16日}
\context{一张兔子的艺术照是抓住兔子拍的}
\lml{这就是抓拍}

\subsection*{1月15日}
\jgl{将会被帅死,帅会被将死}
\jgl{将死之人获得了比赛胜利(}

\subsection*{1月16日}
\dlr{赤弦和余弦有什么区别}
\cyl{@丹尼鹿 \ 赤是正色之一}
\cyl{所以赤弦是一种正弦()}

\subsection*{1月17日}
\jgl{话说}
\jgl{我发现}
\jgl{人行道是人步行道}

\subsection*{1月18日}
\context{赤弦发现FC的万豪酒店的灯的M灭了一半}
\cxr{所以到底什么是λarriott}
\jmr{larriott??}
\npl{@晴天赤弦 \ 五千豪}

\subsection*{1月19日}
\context{赤弦邀请龙羽(夏目龙羽/小毛龙)加入了群聊}
\qsr{xmly}
\cxr{xml.xml}
\xml{其实是}
\xml{xml.yaml}
\xml{now you know Y}


