% main.tex
\documentclass[a4paper, 9pt, twoside]{article}
\usepackage[inner=0.2in, outer=0.2in, top=0.2in, bottom=0.6in]{geometry}
\geometry{papersize={99mm, 210mm}}        % 定义 DL 尺寸

% 字体设置

\usepackage[fontset=none]{ctex}

\setmainfont{NotoSans-VariableFont_wdth,wght.ttf}[
  Renderer=HarfBuzz,
  Path=fonts/, % 末尾保留斜杠
  UprightFeatures    = {Weight=400},  % Regular
  BoldFont           = {NotoSans-VariableFont_wdth,wght.ttf},
  BoldFeatures       = {Weight=700},  % Bold
  ItalicFont         = {NotoSans-Italic-VariableFont_wdth,wght.ttf},
  ItalicFeatures     = {Weight=400}, 
  BoldItalicFont     = {NotoSans-Italic-VariableFont_wdth,wght.ttf},
  BoldItalicFeatures = {Weight=700}
]

\setCJKmainfont{NotoSansSC-VariableFont_wght.ttf}[
  Renderer=HarfBuzz,
  Path=fonts/, % 末尾保留斜杠
  UprightFeatures    = {Weight=400},  % Regular
  BoldFont           = {NotoSansSC-VariableFont_wght.ttf},
  BoldFeatures       = {Weight=700},  % Bold
]

\usepackage{fontspec} % Enables font customization

\newfontfamily\emojifont{NotoColorEmoji-noflags.ttf}[
  Renderer=HarfBuzz,
  Path = fonts/, % 若字体放在 fonts/ 目录下,否则可省略
  UprightFeatures = {RawFeature={+colr;+emoji;palette=0}}, % 启用彩色渲染
  Extension = .ttf,
]

\newfontfamily\Cans{NotoSansCanadianAboriginal-VariableFont_wght.ttf}[
  Renderer=HarfBuzz,
  Path = fonts/,
  UprightFeatures = {Weight=400},
  BoldFont = {NotoSansCanadianAboriginal-VariableFont_wght.ttf},
  BoldFeatures = {Weight=700},
  Scale = MatchLowercase,
]

\newfontfamily\mathfont{NotoSansMath-Regular.ttf}[
  Renderer=HarfBuzz,
  Path = fonts/,
  Extension = .ttf,
]

% 标题设置

\usepackage{titlesec}

\titleformat{\section}
  {\vspace{-1.75ex}\centering\Large\bfseries}
  {}
  {0em}
  {}

\titleformat{name=\subsection,numberless}[block]%
  {% Format settings for subsection
    \needspace{2.5\baselineskip}%  <-- Ensures appropiate lines of space remain
    \vspace{-1.75ex}    % Original style
  }%
  {}             % No label
  {0em}          % No indentation
  {}             % Nothing before subsection title text
  [\hrule height 0.5pt \vspace{0ex}] % After subsection

% 目录设置

\usepackage{tocloft} % 控制目录样式

\renewcommand{\contentsname}{} % 清空目录标题
\setlength{\cftbeforetoctitleskip}{0pt} % 移除标题前的空白
\setlength{\cftaftertoctitleskip}{0pt}  % 移除标题后的空白
\setcounter{tocdepth}{1} % 显示到 \section 层级

% 超链接设置

\usepackage{hyperref}

\hypersetup{
    colorlinks=true, %set true if you want colored links
    linktoc=all,     %set to all if you want both sections and subsections linked
    linkcolor=black, %choose some color if you want links to stand out
}

% 聊天记录样式

\usepackage{chat-history}



% ===============正文===============

\begin{document}

% 定义参与者
% 逗哏的人(出现在左侧)
%2024
\newchatperson{jgl}{avatar/jg.jpg}{花椒果汁}{olive}
\newchatperson{sml}{avatar/sm.jpg}{Simone}{RedOrange}
\newchatperson{dnl}{avatar/dn.jpg}{Dan}{Gray}
\newchatperson{nwl}{avatar/nw.jpg}{星光立交与星河路互通匝道}{RoyalBlue}
\newchatperson{lil}{avatar/li.jpg}{雷狼狼}{Yellow}
\newchatperson{shl}{avatar/sh.jpg}{舒扒皮}{Red}
\newchatperson{aml}{avatar/am.jpg}{阿猫的猫}{Blue}
\newchatperson{bdl}{avatar/bd.jpg}{白貂Akirua}{Goldenrod}
\newchatperson{xfl}{avatar/xf.jpg}{雪风:欢迎来到现实的荒漠}{Red}
\newchatperson{axl}{avatar/ax.jpg}{阿寻}{OrangeRed}
\newchatperson{xjl}{avatar/xj.jpg}{小郊狼}{Tan}
\newchatperson{col}{avatar/co.jpg}{乙酰CoA}{Cyan}
\newchatperson{wxl}{avatar/wx.jpg}{万相}{Gray}
\newchatperson{lrl}{avatar/lr.jpg}{雷尔想吃电弧星}{ProcessBlue}
\newchatperson{yyl}{avatar/yy.jpg}{热烈祝贺发表核心期刊零篇}{Peach}
\newchatperson{jml}{avatar/jm.jpg}{Jelly\_Aries金羊节牧}{Apricot}
\newchatperson{nll}{avatar/nl.jpg}{纳尔}{OrangeRed}
\newchatperson{ehl}{avatar/eh.jpg}{Husky Wang}{CadetBlue}
\newchatperson{yxl}{avatar/yx.jpg}{夜星}{Periwinkle}
\newchatperson{bql}{avatar/bq.jpg}{大镇做不出题家{\emojifont🦌}}{LimeGreen}
\newchatperson{gfl}{avatar/gf.jpg}{光锋-ZH9c418}{Cerulean}
% 2025
\newchatperson{lsl}{avatar/ls.jpg}{莱森}{YellowOrange}
\newchatperson{asl}{avatar/as.jpg}{奥斯虎}{Gray}
\newchatperson{jjl}{avatar/jj.jpg}{金桔}{Yellow}
\newchatperson{xzl}{avatar/xz.jpg}{棕汪}{Dandelion}
\newchatperson{lml}{avatar/lm.jpg}{Night Aurora}{Yellow}
\newchatperson{cyl}{avatar/cy.jpg}{吃鱼的猫}{GreenYellow}
\newchatperson{npl}{avatar/np.jpg}{Np}{Aquamarine}
\newchatperson{xml}{avatar/xm.jpg}{龙羽}{BlueGreen}
\newchatperson{mkl}{avatar/mk.jpg}{Mark}{Goldenrod}
\newchatperson{qsl}{avatar/qs.jpg}{棋山}{Bittersweet}
\newchatperson{hgl}{avatar/hg.jpg}{汉广}{Purple}
\newchatperson{lyl}{avatar/ly.jpg}{落雨}{lightgray}
\newchatperson{yrl}{avatar/yr.jpg}{薏仁}{SeaGreen}
\newchatperson{nhl}{avatar/nh.jpg}{诺禾}{SkyBlue}
\newchatperson{ltl}{avatar/lt.jpg}{鹿-θ}{CornflowerBlue}
\newchatperson{jal}{avatar/ja.jpg}{杰}{BurntOrange}
\newchatperson{cxl}{avatar/cx.jpg}{晴天赤弦}{Orange} % TODO: change first occurance

% 捧哏的人(出现在右侧)
% 2024
\newchathost{cxr}{avatar/cx.jpg}{晴天赤弦}{Orange}
\newchathost{wmr}{avatar/wm.jpg}{{\emojifont🐾}神前 \ 海/彦烈{\emojifont🐾}}{ProcessBlue}
\newchathost{bdr}{avatar/bd.jpg}{白貂Akirua}{Goldenrod}
\newchathost{nlr}{avatar/nl.jpg}{纳尔}{OrangeRed}
\newchathost{gfr}{avatar/gf.jpg}{光锋-ZH9c418}{Cerulean}
\newchathost{lrr}{avatar/lr.jpg}{雷尔想吃电弧星}{ProcessBlue}
\newchathost{hqr}{avatar/hq.jpg}{草莓绿茶曲奇}{Green}
\newchathost{amr}{avatar/am.jpg}{阿猫的猫}{Blue}
\newchathost{xzr}{avatar/xz.jpg}{棕汪}{Dandelion}
\newchathost{yyr}{avatar/yy.jpg}{热烈祝贺发表核心期刊零篇}{Peach}
\newchathost{jmr}{avatar/jm.jpg}{Jelly\_Aries金羊节牧}{Apricot}
\newchathost{ltr}{avatar/lt.jpg}{鹿-θ}{CornflowerBlue}
\newchathost{bqr}{avatar/bq.jpg}{大镇做不出题家{\emojifont🦌}}{LimeGreen}
\newchathost{dnr}{avatar/dn.jpg}{Dan}{Gray}
\newchathost{xfr}{avatar/xf.jpg}{雪风:欢迎来到现实的荒漠}{Red}
\newchathost{nwr}{avatar/nw.jpg}{星光立交与星河路互通匝道}{RoyalBlue}
\newchathost{ehr}{avatar/eh.jpg}{Husky Wang}{CadetBlue}
% 2025
\newchathost{lmr}{avatar/lm.jpg}{Night Aurora}{Yellow}
\newchathost{dlr}{avatar/dl.jpg}{丹尼鹿}{BurntOrange}
\newchathost{qsr}{avatar/qs.jpg}{棋山}{Bittersweet}
\newchathost{lpr}{avatar/lp.jpg}{陆佩}{JungleGreen}
\newchathost{lsr}{avatar/ls.jpg}{莱森}{YellowOrange}
\newchathost{mkr}{avatar/mk.jpg}{Mark}{Goldenrod}
\newchathost{yxr}{avatar/yx.jpg}{夜星}{Periwinkle}
\newchathost{wxr}{avatar/wx.jpg}{万相}{Gray}
\newchathost{rxr}{avatar/rx.jpg}{迹犾}{SkyBlue}
\newchathost{qxr}{avatar/qx.jpg}{一罐儿七喜}{brown}
\newchathost{alr}{avatar/al.jpg}{\Cans{ᐱ ᒪ ᕮ ᙭}}{CadetBlue}
\newchathost{zer}{avatar/ze.jpg}{ZeroDaySober}{Purple}
\newchathost{qer}{avatar/qe.jpg}{乔尔}{Dandelion}


% ===============封面===============

\pagenumbering{gobble} % 禁用页码

\begin{titlepage}
  \centering
  \vspace*{2cm}
  \begin{tikzpicture}[baseline=(current bounding box.north)]
    \clip (0,0) circle(2cm); % Clip the avatar image to a circle
    \node[anchor=center,inner sep=0pt,minimum size=1cm] (avatar) at (0,0)
    {\includegraphics[width=4cm,height=4cm,keepaspectratio]{avatar/group.jpg}};
  \end{tikzpicture}\\[1cm]
  {\Huge\bfseries ``弦狼实验室''}\\[0.5cm]
  {\Huge\bfseries 群年度冷笑话大全}

  \vfill
  \textperiodcentered~2025~\textperiodcentered \\
  {\small (截至\today)}
  % {\small (2024年全)}
\end{titlepage}

\pagenumbering{roman} % 页码重置为阿拉伯数字
\setcounter{page}{1}   % 从第1页开始编号

% ===============目录===============

\section{目录}
% NOTE: 为了将页数凑成4的倍数 目录可以隐藏
\tableofcontents % 添加目录

\newpage

% ===============前言===============

% 参与者头像合影
\section{主要贡献者}
{\vspace{-2ex} \centering \small(2024年起, 按首次收录时间排序) \par}
\hspace{1em}

% 2024
\begin{tikzpicture}[baseline=(current bounding box.north)]
  \avatarwithtext{0}{0}{avatar/jg.jpg}{花椒果汁}
  \avatarwithtext{0}{-0.8cm}{avatar/sm.jpg}{Simone}
  \avatarwithtext{0}{-1.6cm}{avatar/dn.jpg}{Dan}
  \avatarwithtext{0}{-2.4cm}{avatar/nw.jpg}{星光立交与星河路互通匝道}
  \avatarwithtext{0}{-3.2cm}{avatar/li.jpg}{雷狼狼}
  \avatarwithtext{0}{-4.0cm}{avatar/sh.jpg}{舒扒皮}
  \avatarwithtext{0}{-4.8cm}{avatar/am.jpg}{阿猫的猫}
  \avatarwithtext{0}{-5.6cm}{avatar/bd.jpg}{白貂Akirua}
  \avatarwithtext{0}{-6.4cm}{avatar/xf.jpg}{雪风:欢迎来到现实的荒漠}
  \avatarwithtext{0}{-7.2cm}{avatar/ax.jpg}{阿寻}
  \avatarwithtext{0}{-8.0cm}{avatar/xj.jpg}{小郊狼}
  \avatarwithtext{0}{-8.8cm}{avatar/co.jpg}{乙酰CoA}
  \avatarwithtext{0}{-9.6cm}{avatar/wx.jpg}{万相}
  \avatarwithtext{0}{-10.4cm}{avatar/lr.jpg}{雷尔想吃电弧星}
  \avatarwithtext{0}{-11.2cm}{avatar/yy.jpg}{热烈祝贺发表核心期刊零篇}
  \avatarwithtext{0}{-12.0cm}{avatar/jm.jpg}{Jelly\_Aries金羊节牧}
  \avatarwithtext{0}{-12.8cm}{avatar/nl.jpg}{纳尔}
  \avatarwithtext{0}{-13.6cm}{avatar/eh.jpg}{Husky Wang}
  \avatarwithtext{0}{-14.4cm}{avatar/yx.jpg}{夜星}
  \avatarwithtext{0}{-15.2cm}{avatar/bq.jpg}{半群semigroup{\emojifont🦌}}
  \avatarwithtext{0}{-16.0cm}{avatar/gf.jpg}{光锋-ZH9c418}
\end{tikzpicture}

\newpage

%2025 
\begin{tikzpicture}[baseline=(current bounding box.north)]
  \avatarwithtext{0}{0}{avatar/ls.jpg}{莱森}
  \avatarwithtext{0}{-0.8cm}{avatar/as.jpg}{奥斯虎}
  \avatarwithtext{0}{-1.6cm}{avatar/jj.jpg}{金桔}
  \avatarwithtext{0}{-2.4cm}{avatar/xz.jpg}{棕汪}
  \avatarwithtext{0}{-3.2cm}{avatar/lm.jpg}{Night Aurora}
  \avatarwithtext{0}{-4.0cm}{avatar/cy.jpg}{吃鱼的猫}
  \avatarwithtext{0}{-4.8cm}{avatar/np.jpg}{Np}
  \avatarwithtext{0}{-5.6cm}{avatar/xm.jpg}{龙羽}
  \avatarwithtext{0}{-6.4cm}{avatar/mk.jpg}{Mark}
  \avatarwithtext{0}{-7.2cm}{avatar/qs.jpg}{棋山}
  \avatarwithtext{0}{-8.0cm}{avatar/hg.jpg}{汉广}
  \avatarwithtext{0}{-8.8cm}{avatar/ly.jpg}{落雨}
  \avatarwithtext{0}{-9.6cm}{avatar/yr.jpg}{薏仁}
  \avatarwithtext{0}{-10.4cm}{avatar/nh.jpg}{诺禾}
  \avatarwithtext{0}{-11.2cm}{avatar/lt.jpg}{鹿-$\theta$}
  \avatarwithtext{0}{-12.0cm}{avatar/ja.jpg}{杰}
  \avatarwithtext{0}{-12.8cm}{avatar/cx.jpg}{晴天赤弦} % TODO: change first occurance
\end{tikzpicture}

\newpage

% ===============正文===============

\pagenumbering{arabic} % 页码重置为阿拉伯数字
\setcounter{page}{1}   % 从第1页开始编号

% % 2024.tex 

\section{2024}

\subsection*{5月16日}
\jgl{话说人摸电线会触电,那么电线被人摸就会触手?}

\subsection*{5月17日}
\jgl{为什么虹桥机场不能去阿斯加德?}

\subsection*{6月3日}
\context{在讨论JWST不能观测全部的可见光波长}
\cxr{主要是观测红外 \ 从600纳米到25微米}
\jgl{@晴天赤弦 \ 不可见光的东西,要好好查查他们的账本}

\subsection*{6月7日}
\context{在讨论美国光学望远镜建造部门的反间谍措施}
\jgl{蠢人容易被情报人员利用,是因为intelligence被偷走了吗}

\subsection*{6月15日}
\jgl{如果湘菜里有芹菜,那秦菜里有香菜吗?}

\subsection*{6月20日}
\context{在讨论飞机上生的小孩该起什么名字}
\jgl{“你是天生的” “天生什么的?” “就是天生的”}

\subsection*{6月27日}
\context{赤弦误入了几个天文系教授的聊天后错过了开溜时机}
\cxr{不过真要这么说的话我觉得他们应该是在劝退}
\cxr{一直在讲学术圈的黑暗}
\jgl{(毕竟有阳光照到的地方研究不了天文}

\subsection*{6月30日}
\jgl{大部分人都用右手写字是不是因为这才是right hand?}

\subsection*{7月4日}
\context{在讨论购买键盘时的考虑事项}
\wmr{糖进键盘, WD40和酒精都救不了}
\jgl{@{\emojifont🐾}神前 \ 海/彦烈{\emojifont🐾} \ 所以不要使用糖特别多的语言}

\subsection*{7月8日}
\context{一张小英雄其实很强的梗图}
\jgl{@晴天赤弦 \ 小英雄不是触手可得的}

\subsection*{7月9日}
\bdr{为什么黄石公园不卖硫磺皂周边}
\jgl{黄石坐高铁去武汉,然后坐地铁去柏林,这是美国到欧洲最快的路线}

\subsection*{7月11日}
\context{在讨论山东宾果游戏中的刻板印象}
\nlr{我172}
\cxr{可窝认识的山东人均大汉}
\jgl{@晴天赤弦 \ 陕西人均大唐?}

\subsection*{7月19日}
\context{在隔壁群讨论ArXiv中Cosmology分类}
\cxr{好奇怪}
\cxr{用google app打开arxiv 朗读 结果读出来的是8天前的页面的内容}
\jgl{研究cosmology的人叫coser吗?}

\subsection*{7月21日}
\context{在讨论未来规划}
\cxr{窝开卷是为了以后也能像现在这样到处玩}
\sml{对cxjj来说就没有闭卷考试,全都是开卷}

\subsection*{7月21日}
\context{在讨论推特上偶遇的天文学小动物的研究方向}
\cxr{他觉得mit是个好地方}
\cxr{那他肯定是做系外行星的 猜中了.jpg}
\jgl{@晴天赤弦 \ 好厉害,随手搓个行星}

\subsection*{7月22日}
\jgl{愚公移山是个人行为,所以是于私移山}

\subsection*{7月23日}
\jgl{话说……冒脑花是不是一种烫头}
\jgl{只不过烫的是里面}

\subsection*{7月23日}
\context{知乎问题``看完奥本海默突然想学理论物理可行吗''}
\gfr{@晴天赤弦 \ 看完黑客帝国我想学功夫躲子弹可以吗}
\jgl{@光锋-ZH9c418 \ 不不不,那是kongfu不是功夫,意思是你学完只能饿肚子喝西北风}

\subsection*{7月28日}
\jgl{地铁和公交车相比最大的缺点是,它不巴适}

\subsection*{7月30日}
\context{知乎问题``空间和时间是有质量的吗?''}
\jgl{时间当然是有质量的,物理学家会把质量小的年份叫做light year。}

\subsection*{8月1日}
\dnl{如果基督教里666代表邪恶 是不是意味着25.8是邪恶的根}

\subsection*{8月5日}
\context{看到一个好看的画师}
\cxr{想约}
\jgl{@晴天赤弦 \ 为什么要约荤的呢,因为素的不可约}

\subsection*{8月5日}
\context{在讨论雷尔做赤弦毛替的动作和本体出毛不太像}
\cxr{被tf的不够深度}
\lrr{来点深度tf}
\jgl{@雷尔想吃电弧星 \ 深度tf就是cuda?}
\jgl{不对tf本来就是用来深度的啊}
\jgl{transfur的话,也是一种homomorphism}
\jgl{额异性不算}
\jgl{至少确实morph了}

\subsection*{8月10日}
\context{一张内容为``我不管,取我虎符号令百万雄师''的梗图}
\jgl{为啥虎符可以号令雄狮?}

\subsection*{8月11日}
\jgl{为什么找零都是大于零元才找?}

\subsection*{8月17日}
\context{赤弦在发午饭的照片}
\cxr{大块肋眼牛排!}
\jgl{@晴天赤弦 \ 只有刑天才有真正的肋眼}

\subsection*{8月18日}
\jgl{凭啥四川火锅不算有机食品}
\jgl{涮的都是各种内脏}
\jgl{各种内脏下水难道不organic吗?}
\nwl{草}
\nwl{你这有机火锅还是无肌啊}

\subsection*{8月19日}
\jgl{过敏性肤质会麸质过敏吗?}

\subsection*{8月19日}
\lil{“我今年实岁26,虚岁28”王警官说着在蛋糕上点了(26+28i)根蜡烛}
\hqr{@雷狼狼 其实这个还是比较容易}
\hqr{因为蛋糕和C是同构的}
\hqr{只要位置找对}
\jgl{这叫同胚,同的是蛋糕胚}

\subsection*{8月20日}
\context{赤弦发了一张猞猁标本的照片}
\jgl{@晴天赤弦 \ 很多仙侠小说人名喜欢叫啥啥子}
\jgl{那猞猁成精了就是猞猁子}

\subsection*{8月20日}
\context{在讨论各地小学的师资}
\xzr{@阿猫的猫 \ 不是的,至少那个小孩的老师还没我这个县城出身的小学老师讲的明白}
\cxr{班主任天天讲什么 剖腹产出来的学生就是不行}
\aml{是不是减负导致的}
\shl{@阿猫的猫 \ 是剪腹导致的}

\subsection*{8月26日}
\jgl{给没心没肺的人做心肺复苏有用吗?}

\subsection*{8月26日}
\context{赤弦在从晒了一天的车里往外搬柜子}
\cxr{这几个柜子被拷到烫手的根本拿不起来}
\aml{@晴天赤弦 \ 用锟斤拷拿出来}

\subsection*{8月27日}
\jgl{二郎神是不是因为封号太多要念很久,所以叫贯口大神?}

\subsection*{8月29日}
\context{赤弦在感慨人生中的错过}
\bdl{你的人生一错再错}
\bdl{所以你应该和我去西藏}
\bdl{去看纳木错、羊卓雍错、玛旁雍措、拉姆拉措…}

\subsection*{8月31日}
\jgl{燃气灶是不是就是厨师机?}

\subsection*{9月6日}
\jgl{和尚偷懒睡回笼觉,是不是一种叫回龙观的修行?}

\subsection*{9月6日}
\context{在讨论搬到山里后就会变得喜欢发风景照}
\xfl{可是}
\xfl{我从小就住在山里}
\xfl{初中直接沿山而建,体育课项目都是爬山}
\xfl{后面知道这山也是风景区}
\xfl{所以我是爬山狐}

\subsection*{9月8日}
\jgl{鞋店的货架是不是就是脚型架?}

\subsection*{9月9日}
\jgl{话说演员找别人配音是不是一种借口?}

\subsection*{9月9日}
\context{在讨论智利和美国的时差}
\bdr{想不到吧,智利是他妈东部时间}
\bdr{为了方便和巴西阿根廷交流改的,离谱的很啊}
\aml{@白貂Akirua \ 智利障碍}

\subsection*{9月10日}
\cxr{怎么老板会议室里全是清华的贴纸}
\sml{@晴天赤弦 \ 群青}

\subsection*{9月10日}
\cxr{说起来之前去陨石坑玩的时候看到了一对犹太人}
\axl{他们也算陨石么}

\subsection*{9月27日}
\jgl{经常把天聊死的人,是不是会觉得上帝已死?}

\subsection*{9月27日}
\jgl{-鱼生会不会有寄生虫啊?\\-子非鱼,为什么要怀疑鱼生?}

\subsection*{10月2日}
\context{一张显示黑神话悟空的日活完美对应调休的图表}
\wmr{草}
\wmr{黑神话算是创造历史了}
\xfl{所以是黑历史吗}

\subsection*{10月3日}
\context{赤弦在抱怨每天早上7点开除草机扰民的白男}
\bdr{我觉得早上九点之前除草的人真的很rude}
\cxr{@白貂Akirua \ 为什么他们非得那么早除草啊}
\cxr{下午又不是不能除}
\xjl{因为除了草就是早了}
\xjl{所以得早上除}

\subsection*{10月9日}
\jgl{物是人非是不是指东西很好但是抽卡抽不到?}

\subsection*{10月12日}
\jgl{火控好的舰船是不是食堂更好吃?}
\jgl{毕竟火候控制更精准}

\subsection*{10月13日}
\context{赤弦在野外天文观测时担心野生动物钻进车里}
\yyr{睡着睡着钻进来个大母狼福瑞}
\sml{于是赤弦一高兴就把大母狼捂在被窝里了,人称捂大狼}

\subsection*{10月13日}
\sml{你知道有人穿着狼兽装跳舞叫什么吗?叫毛里小舞狼}

\subsection*{11月8日}
\jgl{听说数学家很关心社会现象,他们喜欢研究inequalities,还喜欢代入}

\subsection*{11月18日}
\cxr{ic}
\jgl{@晴天赤弦 显卡也是ic产业,因为通过显卡,ic}

\subsection*{11月22日}
\aml{Q: 为什么妓女不怕IRS?}
\aml{因为是睡后收入}

\subsection*{11月27日}
\jgl{话说军队炊事班地位是不是很高?打仗都要抢着制糕点}

\subsection*{11月30日}
\jgl{面具会不会造成覆面影响?}
\col{@花椒果汁 \ 出租车出租的其实是司机}

\subsection*{11月30日}
\jgl{太空,有空间站。\\太挤,就没有空间站。}

\subsection*{11月30日}
\jgl{足球是不是一种陆游?就像打牌是一种桌游}
\jmr{皆若空游无所依}
\jmr{和我打羽毛球}
\ltr{球场是吧}
\ltr{请给出羽毛球场的场方程}
\jgl{商场是哪种意义上的商?}
\bqr{开一个商场叫商空间}
\cxr{well如果从字本义来说的话}
\cxr{应该是商朝的场}
\jmr{那集体农场的场方程和商场的场方程有同构吗}

\subsection*{12月3日}
\context{Dan分享刚拍的一张绿色偏色严重的星云照片}
\dnr{天使星云拍翻车了、}
\jgl{好专业,怕星星拍出来看不清还在背后加绿幕}

\subsection*{12月4日}
\context{赤弦在分析电脑卡顿的原因}
\cxr{啊啊果然}
\cxr{一打开美卡论坛就开始卡}
\jgl{毕竟是卡论坛}

\subsection*{12月5日}
\jgl{话说酒驾是一种羟基案件吗?}

\subsection*{12月7日}
\context{一张``成都的圣诞树是弯的''的截图}
\wxl{还是被铃压弯的}

\subsection*{12月13日}
\context{在讨论中餐的历史与国际认可度}
\jgl{中餐的劣势是它不是大餐}

\subsection*{12月14日}
\jgl{南大学生也可以是女大学生是不是一种lgbt?}

\subsection*{12月15日}
\sml{狗狗通过撒尿标记地址,是不是也是一种I Pee Address?}

\subsection*{12月15日}
\context{赤弦被火警吵醒后想继续睡接上之前的梦}
\lrl{如果睡前故事和睡后故事完全一样的话,算不算偷睡漏睡}

\subsection*{12月16日}
\context{赤弦在研究怎么在LaTeX中显示emoji}
\jgl{晦月魔君是emoji吗}
\jgl{我是说,恶魔鸡}

\subsection*{12月16日}
\context{赤弦在整理回国行李}
\cxr{啊对去美国还要拿西装}
\jgl{@晴天赤弦 \ 美式兽装不是西装吗?}

\subsection*{12月17日}
\jgl{让孙悟空去当弼马温,正是因为看出他有二心,才让他去做一个stable的工作}
\lrr{那孙悟空当时给自己起名叫drop database name,整个天庭的考勤系统不都得瘫痪}

\subsection*{12月18日}
\jgl{如果你就差一道填空就及格了,那这个答案是password}

\subsection*{12月18日}
\context{合订本征集修改意见}
\cxr{大家也再看看有没有啥问题 没有的话窝就找人去打印装订惹}
\nwl{@晴天赤弦 \ 我觉得}
\nwl{赤弦比较适合去地窝堡机场}
\nwl{赤弦的嘴里总能吐出窝字}
\nwl{还有惹}
\nwl{是不是一种萨拉惹窝运动}
\yyl{冬天睡觉冷可以做一盆沙拉倒进被窝里}
\yyl{这就叫做沙拉热窝}
\yyl{沙拉倒进被窝听见巨大响动是正常的}
\yyl{这就是沙拉热窝的枪声}

\subsection*{12月18日}
\context{合订本征集版式意见}
\cxr{现在这样就是很素的}
\jgl{@晴天赤弦 \ 毕竟也没荤段子}

\subsection*{12月18日}
\context{合订本征集修改意见}
\xfr{指的是群刊做成月历}
\jgl{@雪风:欢迎来到现实的荒漠 \ 熟悉历史可以丰富月历}

\subsection*{12月18日}
\jgl{绝地武士导师指着黑板上的4↑↑↑4说:“This is the power of 4s”}
\jgl{恶魔猎人维吉尔看着i\texttt{\string^}2=1陷入沉思,然后说:“i need more power”}
\jgl{论文里在一句话结尾右上角角标写个[1],意思就是this work is powered by [1]}

\subsection*{12月18日}
\context{在讨论去哪里打印合订本}
\cxr{那窝上飞机前努努力把末班车加上}
\nwr{如果落地了还没淘宝}
\nwr{那时候再考虑就地找广告公司}
\jgl{落地了就上不了榜了}

\subsection*{12月18日}
\nwl{强扭的瓜不甜}
\nwl{因为不带梗}

\subsection*{12月18日}
\jgl{嫦娥一直住在广寒宫,不会宫寒吗?}

\subsection*{12月19日}
\jgl{话说国安是不是特别讨厌artificial intelligence?}
\jgl{而日本的情报学就特别擅长这个}

\subsection*{12月19日}
\jgl{悬丝诊脉是不是最早的线上医疗技术?}

\subsection*{12月19日}
\ehr{话说上次有个来上海的土生土长的奥斯陆上海人,他只会说英文 挪威话特上海话,然后他来上海的原因是去交大学中文}
\jgl{那他中文肯定很好,可以教大学中文}

\subsection*{12月19日}
\context{赤弦在感慨和美国相比上海物价真便宜}
\cxr{窝在美国大部分时间都在吃速冻的东西}
\sml{所以你是管理员,每天都吃sudo的东西}

\subsection*{12月19日}
\context{赤弦在感慨选择总伴随着放弃}
\cxr{早知道会是这样}
\cxr{窝那会要么就开卷要么就开摆了}
\sml{结果又摆又卷}
\sml{最终交出了一份摆卷}

\subsection*{12月20日}
\cxr{话说之前跟印度人聊天知道了一个挺有意思的事情}
\cxr{不要对一个印度人说coconut}
\cxr{因为这个词的意思是棕皮白心}
\cxr{印度居然也有香蕉人的同义词}
\jgl{从范畴的意义上讲,coconut不就是 nut}
\jgl{就像coauthor等于reader}

\subsection*{12月20日}
\jgl{为什么无线充电还有电量限制?}

\subsection*{12月20日}
\jgl{脱毛和卸装是不是一个意思?}
\jml{我是民科,这就是冷硝化}

\subsection*{12月21日}
\context{赤弦发了一张梗图}
\cxr{所有的人事物都会骗你, \\但披萨不会, \\因为, \\披萨只有6片8片12片, \\没有7片.}
\nwl{@晴天赤弦 \ six eight twelve slices, no four(foul fool) slices}

\subsection*{12月21日}
\context{大家拼Daniel的车去吃饭}
\lrl{daniel的车坐满了之后,是不是可以叫共享dan车}

\subsection*{12月21日}
\lrl{沪国的沪照是护照吗}
\nll{北京的叶子是京叶吗}
\nll{苏州的考试是苏轼吗}

\subsection*{12月22日}
\context{大家和赤弦的FURRY车牌合影}
\ehl{拍牌照.jpg}

\subsection*{12月22日}
\ehl{洗过脚的水叫洗脚水,那洗过电路板的水是不是叫洗板水}
\ehl{但是洗过米的水为什么叫淘米水}
\lrl{@Husky Wang \ 因为赛尔号找到了无尽能源,拯救了地球}

\subsection*{12月24日}
\nll{食人族抓了一个中国小孩,说:今晚吃汉宝宝}

\subsection*{12月24日}
\context{No One but Nemo发了一张``冻停狐''的梗图}
\jml{倍加二狐}
\yyl{冰塘雪狸}
\aml{钉冬鸡}

\subsection*{12月24日}
\jml{最昂贵也是最廉价的一个afternoon}
\jml{good afternoon}

\subsection*{12月24日}
\jml{今天发烧了}
\jml{群友吃羊就不用加调料了}
\jml{因为我加了自燃}

\subsection*{12月25日}
\jml{在床头挂袜子算不算是一种圣诞钓鱼}
\jml{还愿者上钩}

\subsection*{12月25日}
\lrl{温哥华人算不算华人}
\xfl{印第安人都不是安人}
\yxl{马来西亚人也不是亚人}
\sml{不知道,但浙江金华人应该是华人}
\sml{本人也不是日本人}
\yxl{@夜星 \ 虽然如果有亚人(半兽半人)那种的话好像也不错}

\subsection*{12月26日}
\jgl{为什么x光通常是平行投影,没有透视?}
\bql{可能需要y光和z光}
\jgl{那有y国人为啥没有x国人}

\subsection*{12月27日}
\jml{什么时候可以见赤弦}
\cxr{@Jelly\_Aries金羊节牧 \ 有缘的时候uwu}
\jml{出门需要随身携带一个π}

\subsection*{12月27日}
\context{在讨论用benty-fields订阅ArXiv文章}
\xzr{我试试}
\cxr{@棕汪 最主要的是他有Journal club \ 加入一个club后可以给文章投票} 
\jgl{@晴天赤弦 \ 这是一种棍棒教育吗}
\jgl{各种club}

\subsection*{12月28日}
\lrl{为什么路易十六很色,因为色字头上一把刀}

\subsection*{12月29日}
\context{赤弦在发毛图, 阿猫的猫在意大利旅游}
\aml{赤弦就像罗马的pasta}
\aml{好赤,但是很弦}

\subsection*{12月30日}
\context{一张``普通人每年大概会有127次性生活''的梗图}
\jgl{这样吗?我一直以为人是一次性的,猫是9次性的}

\subsection*{12月31日}
\context{在讨论README.md中md的全称是什么}
% shrink to fit 36 pages
% \wmr{话说.md}
% \wmr{}
% \wmr{markdown?}
% \lrl{嗯}
\lrl{你也可以当做是程序员的怒吼}
\lrl{嘛的!}
\gfl{显然不可能是MaimaiDx}
\xfl{因为有的人不看文档就会攻击程序员,所以md文档实际上是导弹防御的一部分}

% 2025.tex 

\section{2025}

\subsection*{1月1日}
\lil{动物园抽血监测 轮到大象的时候大家都喊我}
\lil{因为我负责抽象}

\subsection*{1月1日}
\jgl{firefighters是不是裁军的意思?}

\subsection*{1月1日}
\context{阿猫的猫想起发MFF的图了}
\amr{草}
\lsl{@阿猫的猫 \ 阿猫的猫草}

\subsection*{1月2日}
\context{在讨论晒车}
\asl{我只能}
\asl{晒太阳}
\nwl{我}
\nwl{车没痛}
\nwl{所以不晒了}
\jml{*晒羊毛*}
\lrl{车太黑了,再晒就要被抓去种棉花了}
\jjl{没事,你可以晒鑫牌太阳}
\asl{学历有水分,所以只需要将其晒干即可}

\subsection*{1月2日}
\context{在讨论小时候的麦当劳味道再也没回来过}
\cxr{窝的话}
\cxr{自从吃过了美国的麦当劳肯德基后}
\cxr{就越发想念中国麦当劳惹}
\xfl{@晴天赤弦 \ 想你了,劳大}

\subsection*{1月2日}
\jgl{- commission closed \\ - may i consider it a complement?}

\subsection*{1月4日}
\context{在讨论赤弦的塔罗牌主题的委托}
\lmr{奖牌和法环好像}
\cxr{其实是塔罗牌的星币}
\jgl{@晴天赤弦 \ 说起来starbucks好像就是星币}

\subsection*{1月5日}
\jgl{情侣约会的地方是不是叫七夕地?因为生育率着急所以要保护七夕地}
\jml{我看这个生育率是有乞巧的}

\subsection*{1月6日\footnote{由于统计时的中美时差, 本文标题记录的日期可能会存在不超过一天的误差.}}
\xfl{今天是2024年的头七}

\subsection*{1月6日}
\dnl{飞机动车速度都那么快,那为什么非机动车道限速才15?}

\subsection*{1月7日}
\asl{维京人会戴围巾吗}

\subsection*{1月7日}
\cxr{busybox是不是一种忙盒}
\jjl{是个恐怖盲盒,拆出了Kernel panic}

\subsection*{1月12日}
\jgl{三亚坑游客钱是不是因为那里都是穷B?}

\subsection*{1月13日}
\context{白貂收到了一封标题写着[FREE FOOD]的会议邮件}
\bdr{很懂嘛}
\xzl{@白貂Akirua \ 当学术大拿,在学术会议上大拿特拿}

\subsection*{1月13日}
\context{赤弦明年去凤凰城开会又会遇上兽展}
\cxr{西雅图是ANW 凤凰城是PDFC}
\bdr{你们去凤凰城有点亏啊}
\bdr{这么近}
\bdr{不开会也能去}
\cxr{@白貂Akirua \ 可以开车去 这不省事多了嘛}
\xzr{@晴天赤弦 \ 不能借机旅游呀}
\xzr{我这几年的旅游都是蹭出差去的}
\sml{@棕汪 \ 借机旅游不如包机旅游}

\subsection*{1月15日}
\jgl{听说日本都是日麻不是川麻,感觉那边麻婆豆腐不正宗}

\subsection*{1月16日}
\context{一张兔子的艺术照是抓住兔子拍的}
\lml{这就是抓拍}

\subsection*{1月15日}
\jgl{将会被帅死,帅会被将死}
\jgl{将死之人获得了比赛胜利(}

\subsection*{1月16日}
\dlr{赤弦和余弦有什么区别}
\cyl{@丹尼鹿 \ 赤是正色之一}
\cyl{所以赤弦是一种正弦()}

\subsection*{1月17日}
\jgl{话说}
\jgl{我发现}
\jgl{人行道是人步行道}

\subsection*{1月18日}
\context{赤弦发现FC的万豪酒店的灯的M灭了一半}
\cxr{所以到底什么是λarriott}
\jmr{larriott??}
\npl{@晴天赤弦 \ 五千豪}

\subsection*{1月19日}
\context{赤弦邀请龙羽(夏目龙羽/小毛龙)加入了群聊}
\qsr{xmly}
\cxr{xml.xml}
\xml{其实是}
\xml{xml.yaml}
\xml{now you know Y}




% ===============封底===============

\newpage
\pagenumbering{gobble} % Disable page numbering

% 空页 用于页数凑整
% \null
% \newpage

\null
\vfill

\begin{quote}
    \centering
    感谢各位群友贡献的聊天记录.
\end{quote}
\begin{quote}
    \centering
    本文由\LaTeX 编写. 访问 \input{ |git remote get-url origin} 获取本文的源代码. 
\end{quote}

\end{document}
